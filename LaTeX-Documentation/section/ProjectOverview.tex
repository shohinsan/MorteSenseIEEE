\section{Project Overview}\label{sec:project-overview}

\subsection{Project Goals and Objectives}\label{subsec:project-goals-and-objectives}
The \brand{} DIY security system, more accurately described as a Modular Microwave Motion
Security System with mobile notifications, was a project intended to provide efficient
and reliable security for homes and businesses. The system uses microwave motion
sensors to detect movement within a certain range and sends a mobile notification to
the owner's phone in case of any suspicious activity.

There was an emphasis placed on the open documentation with the project group working
closely with the advisors to allow for prospective end users to take the system design
and modify it to their own needs and specifications, along with allowing some degree of
freedom to enable end users to add their own modules such as cameras, passive infrared
(PIR) sensors, and piezoelectric sensors. In addition to allowing physical modularity, end users
may also use different software tools and services for the back end of the proposed design for
long-term sustainability and avoid being forced into a single ecosystem.

\subsection{Problem and Motivation}\label{subsec:problem-and-motivation}

The \brand{} project aims to address the growing need for accessible, cost-effective,
and reliable security solutions for residential and commercial properties.
Traditional security systems often come with proprietary modules, firmware, software,
or communication protocols, making them impracticale for end users that desire system flexibility.

The motivation behind \brand{} was to create a DIY security system that leverages
microwave motion detection technology to provide an affordable, easy-to-install, and
efficient building block towards monitoring and securing properties. This project is important as
it can pave the way to make more advanced security systems accessible to a wider
audience, thus improving safety and security for individuals and businesses.

To ensure real-time alerts, the system incorporates mobile notifications (with the Twilio API
being used for demo purposes), which enable swift responses to potential security breaches by
allowing for the use of different means of communication. This approach effectively addresses
the need for a more convenient and user-friendly security solution.

\subsection{Project Application and Impact}\label{subsec:project-application-and-impact}

In the Age of Information, advanced security systems have become crucial in ensuring public
health, safety, and welfare. With growing concerns regarding security, the need for
efficient and reliable security systems is more pressing than ever. This system has been designed
with critical factors such as global, cultural, societal, environmental, and economic
considerations in mind, ensuring accessibility and effectiveness for a wider range of users.

Public health, safety, and welfare are essential considerations for any security system.
The microwave motion security system with mobile notifications can detect and notify homeowners
of potential security risks through different means, which is vital for public safety.
The mobile notification feature allows prompt action to be taken, reducing the risk of
harm and property damage. By reducing crime rates in a neighborhood, the security system
can also positively impact the safety and well-being of the community. In case of a
security breach, the system can notify emergency services which minimizes damage and protects lives.

The repository has been published onto GitHub and has been continuously for clarity and ease of
use with respect to code and related documentation with the help of the San Jose State
University academic community. It can be found in the following citation. \cite{MorteSense-2023}