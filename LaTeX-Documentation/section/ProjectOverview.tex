\section{Project Overview}\label{sec:project-overview}

\subsection{Project Goals and Objectives}\label{subsec:project-goals-and-objectives}
The MorteSense DIY security system, more accurately described as the Microwave Motion
Security System with SMS Notifications, was a project intended to provide efficient
and reliable security for homes and businesses. %
The system uses microwave motion sensors to detect movement within a certain range and sends an SMS notification to the owner's
phone in case of any suspicious activity. %

To manage the development process efficiently, the team followed the Agile Scrum framework. %
The user stories in this project involved creating the sensor unit, communication module,
and SMS notification system. %
The Project requirements included designing an intuitive user interface, ensuring compatibility with different devices, and developing a robust and
secure system that is easy to install and operate. %

\subsection{Problem and Motivation}\label{subsec:problem-and-motivation}

The MorteSense project aims to address the growing need for accessible, cost-effective,
and reliable security solutions for residential and commercial properties. %
Traditional security systems often come with high installation and maintenance costs and complex
setups, making them unattainable or impractical for many people. %

The motivation behind MorteSense was to create a DIY security system that leverages
microwave motion detection technology to provide an affordable, easy-to-install, and
efficient means of monitoring and securing properties. %
This project is important as it can pave the way to make more advanced security systems accessible to a wider
audience, thus improving safety and security for individuals and businesses. %

To ensure real-time alerts, the system incorporates SMS notifications, which enable
swift responses to potential security breaches. %
This approach effectively addresses the need for a more convenient and user-friendly security solution. %

\subsection{Project Application and Impact}\label{subsec:project-application-and-impact}

In today's society, advanced security systems have become crucial in ensuring public
health, safety, and welfare. %
With growing concerns regarding security, the need for efficient and
reliable security systems is more pressing than ever. %
The Microwave Motion Security System with SMS Notifications has gained popularity in recent years
due to its many benefits in preventing harm and property damage. %
This system has been designed with critical factors such as global, cultural, societal, environmental,
and economic considerations in mind, ensuring accessibility and effectiveness for a
wider range of users. %

Public health, safety, and welfare are essential considerations for any security system. %
The microwave motion security system with SMS notifications can detect and notify
homeowners of potential security risks, which is vital for public safety. %
The SMS notification feature allows prompt action to be taken, reducing the risk of harm
and property damage. %
By reducing crime rates in a neighborhood, the security system
can also positively impact the safety and well-being of the community. %
In case of a security breach, the system can notify emergency services which minimizes damage and
protects lives. %

\subsection{Project Results and Deliverables}\label{subsec:project-results-and-deliverables}

\subsubsection{Project Results}

The project results include a microcontroller and microwave motion sensor
enclosed inside a 3D-printed case DIY motion detector, and a deployed User Management Services (UI) in Amazon AWS\@. %

\subsubsection{Deliverables}

The following are the deliverables set during Spring 2023:
\begin{itemize}
    \item High-Level requirements
    \item Software Requirements Specification document
    \item Roadmap
    \subitem This involved a conversion of the provided schedule document from CMPE 195A into a Gantt Chart
    \item Software Backlog
    \subitem Involved documenting different product features and specifications according to the requirements, then assigning different people to implement them
    \item Hardware Backlog
    \subitem Involved documenting the history of making changes to the hardware, and figuring out the best implementations among the different configurations
    \subitem Will mostly fall under internal-use, but was occasionally referenced or drawn upon for project submissions
\end{itemize}

The following are the deliverables set during Summer 2023:
\begin{itemize}
    \item Creation of CAD Specification Document that will outline the
    constraints that the enclosure(s) must follow for the system
    \subitem Iterative revisions of the 3D-printed enclosure(s) will follow
    \subsubitem Likely will be logged in hardware backlog document
    \item A prototype enclosure for the central module is expected at the end of Summer 2023
\end{itemize}

The following are the deliverables set during Fall 2023:
\begin{itemize}
    \item Start of development
    \subitem Involved assembling the initial prototype of the hardware and coding
    the network communication between hardware and software
    \subitem Set up the communication of the microcontroller and motion detector
    \subitem The Backend API was initialized and deployed, Twilio gateway was pushed to Fall 2023
    \item Creation of a CAD Specification document that will outline the constraints
    that the enclosure(s) must follow for the system
    \subitem Iterative revisions of 3D-printed enclosure(s) were done throughout the
    semester
    \subitem Was logged in the hardware backlog and followed previously mentioned rules
    \item Integration with Amazon AWS
    \subitem Integration of the User Management Service (UI) into the backend
    \subitem Deployment and integration of Amazon AWS cloud computing,
    analytics, and reporting services
    \item More Testing and Configuration
    \item Final Deliverables
    \subitem Hardware is enclosed in a 3D-printed case
    \subitem Final Project Presentation
\end{itemize}