\section{System Implementation}\label{sec:system-implementation}

\subsection{Implementation Overview}\label{subsec:implementation-overview}
In the implementation phase, the group employed a range of technologies including MicroPython, GitHub, Vercel, React, Flask, AWS, and TailwindCSS. This phase encompassed
both macOS Sonoma and Windows 10 environments, and it involved various platforms,
programming languages, and dependencies. %

The group gained practical experience in hardware-related tasks, including wiring and
soldering. %

\subsection{Implementation of Developed Solutions}\label{subsec:implementation-of-developed-solutions}

\textbf{Version Control}: Git was used for version control, and GitHub facilitated
collaboration by allowing team members to join as collaborators. %
Feature branches were employed for different project segments, and GitHub Issues and Projects were
utilized to track tasks, bugs, and project progress. %

\textbf{Code Editors (IntelliJ/VS Code)}: The team created and configured the project
within IntelliJ IDEA, taking advantage of its coding and debugging capabilities.
Git and GitHub integration was seamless, enhancing code quality and maintainability
with built-in refactoring tools. %

\textbf{Frontend}: The frontend was developed using reusable React components for scalability. %
Data and component communication were managed through React's state and props. %
Client-side routing was implemented using React Router for a single-page application (SPA) experience. %
Data from the backend was integrated using asynchronous API calls with tools like fetch or
Axios, and component styling was achieved using CSS-in-JS libraries such as Tailwind CSS\@. %

\textbf{Backend}: The team developed APIs with Flask to serve data to the React frontend
through defined routes and endpoints. %
Integration with the MySQL database system was established. %
Secure user access was ensured through authentication and authorization using Flask-Login or Flask-JWT. %
Middleware was utilized for tasks such as logging, error handling, and request/response modification. %
Reliability was guaranteed by writing unit and integration tests for the Flask application. %

\textbf{Deployment}: The React frontend was deployed on web hosting platforms like Vercel,
while the Flask backend was hosted on cloud services like AWS. This involved setting up
the database, configuring environment variables, and implementing security measures. %
Testing and deployment were automated using CI/CD pipelines like GitHub Actions for a
streamlined development workflow. %

\subsection{Implementation Problems, Challenges, and Lessons Learned}\label{subsec:implementation-problems-challenges-and-lessons-learned}

\textbf{Integrating Complexity}: Integrating the frontend (React) with the backend (Flask)
presented challenges, especially due to differences in technologies, data formats, and
API endpoints. %
Ensuring seamless communication and data flow between the two components required thorough API documentation, cross-functional collaboration, robust error handling,
and end-to-end testing to validate data exchange. %

\textbf{Security}: Implementing authentication and authorization correctly was crucial but
complex. %
Ensuring the security of user data against potential vulnerabilities like SQL injection and cross-site scripting (XSS) required ongoing education in security best
practices, stringent input validation and sanitization, the use of well-established
authentication libraries, regular security audits and code reviews, and the implementation
of security headers and policies to mitigate potential risks. %

\textbf{Database Management}: Setting up and managing a database system presented challenges,
particularly with large datasets. %
Challenges included database migrations,schema design, and optimizing database queries. %
Addressing these challenges involved dedicating time to robust schema design, implementing automated database backups, using
migration tools for schema changes, continuous monitoring and optimization of database
queries, and establishing data cleanup routines to maintain database health. %