\section{Background and Related Work}\label{sec:background-and-related-work}

\subsection{Research and Technologies}\label{subsec:research-and-technologies}

In recent years, DIY home security has risen in popularity, with modern solutions including
doorbell cameras, fingerprint-scanning door modules, infrared motion detection sensors, and
the occasional surveillance camera. Companies like Ring, Blink, Wyze, SimpliSafe, and Vivint
have had a significant impact on improving the availability of home security and in the
development of smart homes that are connected to the Internet of Things (IoT).

Leading up to this point, contemporary literature has explored how using different parts
from different manufacturers in unique configurations can be integrated into
IoT~\cite{sarhan2020}. Additionally, current literature explores how DIY home security will
develop in relation to the blockchain~\cite{arifEtAl_2020} and the growing influence
of machine learning\cite{khanEtAl2021}. In light of this related work, this project entailed
creating a DIY security system that will utilize contemporary microcontrollers and microwave
motion detection sensors and will be able to interface with mini-motion proximity cameras.
Modular design principles were used to create an efficient, small-scale security system that
can add different access points through different modules. Combined with extensive
documentation, users are able to test, maintain, and repair the system as needed.

In addition to using modular design principles, modern technologies including using Wi-Fi
to connect to remote servers and notify users of events with SMS text messaging as well
as web protocols to broadcast the history of detections/captures and enact security
network management, allow for an open ecosystem that can be tailored to the user’s
requirements/specifications and remain sustainable by persisting even if direct
manufacturer support is no longer available.

In the context of this project, the DIY security system builds upon the work established
by research conducted by IEEE engineers and scientists that used the Arduino and
Raspberry Pi platform. This project explored cutting-edge devices and technologies that were
not necessarily present at the time of previous studies, and further analysis
is given in the State-of-the-Art section following this literature review.

Komninos et al. explores the threats to smart homes and smart grids with respect to
software security and attacks involving invasions. This study uncovered that smart grids have
the threat of message modifications, replay attacks, metric impersonation, and
denial-of-service, while smart homes have similar threats, with unique ones including
false synchronization and eavesdropping~\cite{komninosEtAl2014}.

The main concerns for current smart homes now include false synchronization and
denial-of-service, being the most convenient methods for burglars to effectively invade and
loot a home. The two aforementioned attacks were taken into account when developing the DIY
security system, learning from the limitations of this particular study.

A 2017 study conducted by two IEEE researchers explored the vulnerabilities that existed
in DIY home security systems that used sensors, microcontrollers, Raspberry Pi, and ZigBee
communications~\cite{joseMalekian2017}. Jose and Malekian found that one limitation of these
systems at the time included the lack of an algorithm used to understand typical
user behavior and tripping an alarm when the algorithm judges the detected behavior as
erratic. Another limitation of systems at the time was that the ZigBee and IEEE 802.15.4
communications protocol were susceptible to replay attacks, allowing attackers familiar
with the technology to exploit the vulnerability and intrude
into the home~\cite{joseMalekian2017}.

Arif et al. examined the development of blockchain technology and its ability to
handle great demand in its use cases. They evaluated the possibility of using blockchain
technology to back IoT devices in DIY smart home security with a proposed blockchain
configuration (see Appendix D). They found that while it was cost-effective and remained
independent of cloud storage, the computation difficulty was set to the lowest possible,
which may become complicated as more devices are added to the security system [4].
Variable computation difficulty is a new technique gaining popularity, but further
research was made on the hardware requirements, and it was found to require more
resources than what had been planned for the scope of the project.
This is because the project will include web services and a deliberately
designed user experience.

IEEE member Qusay Sarhan conducted a comprehensive literature review in 2020, providing an
overview of dozens of research reports on smart home safety and security systems that
used the Arduino platform. It was found that the most significant challenges to building
these systems included physical attacks, device failure (of microcontroller/module),
power/internet outages, and software compatibility~\cite{sarhan2020}. These are factors that
weighed heavily on this project and provided a point of reference for further learning
and improving the work that has already been done in the field.
Additionally, the review provided a few subjects of interest, including extendability,
performance, visualization, and testing for stress/robustness~\cite{sarhan2020}.

Lastly, Khan et al. built on the work conducted by Arif et al. by exploring the role that
machine learning could play in home security systems that utilize blockchain technology. They
found that using a Deep Extreme Learning Machine (DELM) combined with blockchain could prove
to be far more efficient compared to other algorithms~\cite{khanEtAl2021}. Using machine
learning was beyond the scope of this project, but it is a valid consideration in building
upon the work for a graduate research project.

\subsection{State of the Art}\label{subsec:state-of-the-art}

The current state-of-the-art home security systems employ different techniques and products
to create a comprehensive system that can typically be tailored to the end user through the
use of modules. These products are often well-established and are only revised in ways that
are not directly observable by the end user. Rather, the underlying technology may change
while the enclosure receives minor modifications at most~\cite{sarhan2020}. The most common
implementation for state-of-the-art home security systems involves the use of modules that
connect wirelessly to a central unit through some wireless protocol~\cite{joseMalekian2017}.

Alternatively, each individual module can connect to the internet~\cite{sarhan2020}. From
there, the modules are handled by a mobile application that allows the user to configure
notifications for detection, remotely set the alarm based on activity in the household, or
remotely activate the alarm from a safe place outside the house if there is dangerous
activity observed within~\cite{joseMalekian2017}.

Most modules are typically motion sensors or cameras, with motion sensors having had
much more development over the past few years compared to cameras. For the past decade,
motion sensors in DIY home security systems have used Passive Infrared (PIR) technology to
detect motion with changes in temperature induced by moving objects compared to the ambient
environment~\cite{sarhan2020}. However, one limitation of PIR sensors is that they cannot
detect motion past objects, which means that large furniture may prevent detection.

Another limitation of PIR sensors is that their perception of temperature may be thrown off
by high ambient temperature or by wind, meaning that if the weather is hot, it may not detect
motion as a result of a false negative, or if it is windy, it may create a false
positive~\cite{sarhan2020}. In short, PIR sensors have a reduced effective distance because
of the environmental factors that can affect their ability to make accurate detections.
Nonetheless, the latest advancements have resulted in the recent acceptance of microwave
motion detector sensors that utilize the Doppler effect to detect motion behind objects
and are not susceptible to the environmental factors that impact PIR sensors~\cite
{sarhan2020}. This means that cutting-edge DIY home security can more reliably detect motion
behind large furniture and can be installed outside the house without as much concern for
environmental factors affecting the capacity to detect motion.

The Raspberry Pi and Arduino platforms have been extensively used as research tools by the
IEEE to explore different wireless communications technologies in the DIY space, as well as
for a central DIY security platform that is compared to established home/personal security
products~\cite{sarhan2020}. Over the past decade, Arduinos and their respective
microcontrollers have increasingly used different connectivity types such as XBee, Ethernet,
and Wi-Fi, most recently~\cite{sarhan2020}. With the release of the 802.11n-capable Raspberry
Pi Pico W in 2022, the project builds upon the work covered by previous research that used
both mainline platforms or similar microcontrollers.