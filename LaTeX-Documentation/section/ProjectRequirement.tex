\section{Project Requirements}\label{sec:project-requirements}

\subsection{Domain and Business Requirements}\label{subsec:domain-and-business-requirements}
The \brand{} project had several requirements that the development team met to
create an efficient and reliable security system. %
These requirements included the system's ability to detect motion accurately and differentiate between human and
non-human motion to avoid false alarms. %
Optional requirements included distinguishing between different types of human motion, web-based monitoring and control, and a
built-in camera. %
The system's non-functional requirements included being reliable, secure,
easy to use, and having a long lifespan with minimal maintenance. %

\subsection{System Functional Requirements}\label{subsec:system-functional-requirements}

Below are the functional requirements deemed essential for the project:
\begin{itemize}
    \item The system shall be able to detect motion accurately using microwave sensors with a range of at least 10 meters
    \item The system shall be able to send SMS notifications to a specific mobile device within 5 seconds of detecting motion
    \item The system shall have a built-in algorithm to differentiate between human motion and other types of motion, such as pets or moving objects, to avoid false alarms
    \item The system shall have a remote control that allows users to arm and disarm the system easily
    \item The system shall have a backup power supply that can provide at least 24 hours of continuous operation during power outages
\end{itemize}

The only functional requirement that was deemed desirable for the project involves the
system having a web-based interface that allows users to remotely monitor and control the
system, view motion event history, and adjust system settings. %

\subsection{System Nonfunctional Requirements}\label{subsec:system-nonfunctional-requirements}

Below are the nonfunctional requirements deemed essential for this project:
\begin{itemize}
    \item The system shall be reliable and have a low false alarm rate of no more than 1%
    \item The system shall be easy to install and set up with clear instructions and user manuals provided
    \item The system shall be secure and prevent unapproved entery to private data or system configurations
\end{itemize}

Below are the nonfunctional requirements deemed desirable for this project:
\begin{itemize}
    \item The system should have a user-friendly interface that is intuitive and easy to use, with clear visual and auditory feedback provided
    \item The system should be design to have lower power consumption, consuming no more than 5 watts per hour to reduce operating costs
\end{itemize}

\subsection{Context and Interface Requirements}\label{subsec:context-and-interface-requirements}

With respect to interaction design, the team incorporated a user-friendly interface, clear
instructions, and collected feedback from the advisor and cohorts to ensure that the
security system was accessible to a broader audience, including older or less
technologically-savvy individuals, making it more effective and better-suited to the
intended audience. %

\subsection{Technology and Resource Requirements}\label{subsec:technology-and-resource-requirements}

There are several different resource requirements, and have been compiled into three
different categories: Hardware, Software, and Miscellaneous. %

With respect to hardware, the resources used include a Raspberry Pi Pico W Microcontroller
with a 2.4GHz Wi-Fi Module furnishing, a 5.8GHz Microwave Motion Sensor, and the Thonny IDE
with Micropython. %
The Microcontroller is programmed to connect to the home Wi-Fi network and send API requests to the backend server after processing motion detection data.
The motion sensor emits electromagnetic radiation that penetrates through non-solid objects and bounces
off solid objects. %
The Doppler effect is used to detect the slightest movement changes and the sensor makes use of serial communications to exchange data between the microcontroller
and the motion sensor. %
This allows modification of sensitivity, distance, and other parameters of the sensor. %
Thonny allows for compiling of Micropython into C code to effectively program the hardware. %

With respect to software, the resources used include the Python Flask framework for the
database server, ReactJS for the frontend, the Twilio API for notification service,
and Amazon AWS for cloud computing and reporting analysis. %
The Flask framework responds to the provided hardware API requests and provides a web-based interace to allow users
to remotely monitor and control the system. %
The server functions as a gateway between AWS services and a SQLite database to store and provide systems information. %
The Twilio API is responsible for sending SMS notifications to client devices, and Amazon
AWS is responsible for handling and analyzing data traffic to provide different metrics
for what the sensor(s) report. %

In terms of miscellaneous resource requirements, the Fritzing circuit design software,
SolidWorks CAD software, soldering tools, and 3D printing hardware were used to design,
simulate, and improve upon the hardware at different stages of assembly. %
Fritzing was used during ideation to design and simulate the hardware scheme before actual assembly
and soldering. %
SolidWorks was used to design and prototype enclosures for the
component assembly, and was aided by the Makerspaces available at San Jose State University. %