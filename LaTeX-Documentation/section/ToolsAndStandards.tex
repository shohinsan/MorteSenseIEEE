\section{Tools and Standards}\label{sec:tools-and-standards}

\subsection{Tools Used}\label{subsec:tools-used}
Throughout the project, a variety of tools were used in the development process to
ensure efficient collaboration among team members. %
The following tools were important in achieving the project goals:

\textbf{Software}
\begin{itemize}
    \item \textbf{IntelliJ IDEA and Visual Studio Code}: Primary IDEs used for coding,
    debugging, and managing the project
    \item \textbf{Thonny IDE}: Secondary IDE used for coding/debugging hardware
    programming
    \item \textbf{PyCharm IDE}: Secondary IDE
    \item \textbf{MicroPython}: Employed for programming microcontrollers and enabling embedded system functionalities
    \item \textbf{MySQL}: Relational database management system used to keep track of notification and users
    \item \textbf{ReactJS}: Utilized for building the frontend of the application, providing a dynamic and responsive user interface
    \item \textbf{TailwindCSS}: Utilized for efficient and scalable styling of React components, enhancing the user interface design
    \item \textbf{Python Flask}: Chosen as the backend framework to handle data processing and server-side logic
    \item \textbf{Vercel and AWS}: Respectively used for deploying the frontend and backend on web hosts and cloud services
\end{itemize}

\textbf{Hardware}
\begin{itemize}
    \item \textbf{Soldering Kit}: Used for assembling the hardware components in a compact configuration
    \item \textbf{Multimeter}: Applied in testing and troubleshooting electrical circuits and components
    \item \textbf{Oscilloscope}: Used for monitoring and analyzing signals within the circuits
    \item \textbf{SolidWorks}: Designing and printing the hardware enclosure
    \item \textbf{Fritzing}: Designing and simulating the hardware assembly
\end{itemize}

\subsection{Standards}\label{subsec:standards}

During the course of the project, relatively few established standards were used to
define the workflow of the developer team, but there was an emphasis on referencing
documentation as much as possible while creating the documentation of the product. %
Below is a list of the standards used for different aspects of the project:

\textbf{Software}
\begin{itemize}
    \item \textbf{REST (Representational State Transfer)}: A standard for designing
    robust and scalable APIs
    \item \textbf{SHA-2 (Secure Hash Algorithm)}: Algorithm used for securing
    sensitive data of users
    \item \textbf{JWT (JSON Web Tokens)}: Algorithm used for secure transmission of
    information between parties
    \item \textbf{Agile and Scrum Methodologies}: Approaches intended for efficient
    project management and iterative development
    \item \textbf{Python Flask}: Framework for creation and gateway management of API
    requests
    \item \textbf{Twilio API}: Responsible for handling notifications sent to mobile
    devices from sensor, documentation was referenced to set up wireless communication
\end{itemize}

\textbf{Hardware}
\begin{itemize}
    \item \textbf{802.11xx Wi-Fi}: Wireless network standard used by the
    microcontroller
    \item \textbf{Raspberry Pi Pico W}: Documentation was heavily used for pinouts
    and circuit design~\cite{picoW_docs2016}
    \item \textbf{5.8GHz Microwave Motion Sensor}: Documentation was lightly
    referenced for pinout in relation to the microcontroller~\cite{CQRobot_specs2016}
\end{itemize}