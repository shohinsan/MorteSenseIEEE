\section{System Design}\label{sec:system-design}

\subsection{Architecture Design}\label{subsec:architecture-design}

\subsubsection{Hardware Architecture}

% \img{pcbDiagram}{PCB Diagram \cite{CQRobot_specs2016}}

\img{hardwareArchitecture}{Hardware Architecture \cite{CQRobot_specs2016} \cite{picoW_docs2016}}

The PCB will have 3 main components, the Raspberry Pi Pico W Microcontroller with
2.4Ghz Wi-Fi Module which is in the right top corner, the 9V Battery input in the
right bottom corner, and finally, it will have the 5.8Ghz Microwave Motion Sensor
which is in the left top corner of the board. The motion sensor emits electromagnetic waves
which are then reflected back to the receiver and analyzed. If the waves are altered that
means the object that reflected them is moving. The board also will have LED indicators for
showing the WI-FI connection indicator, power indicator, motion detection indicator, and
connection to backend service indicator.

Rx-Tx serial communication was used between the
microcontroller and motion sensor, for all other communications, the GP pins were used. The microcontroller has native Wi-Fi implementation, which means that it will be able to communicate with the server without need for additional modules to enable 802.11 Wi-Fi communications. All the hardware will be programmed in C and Micropython and was programmed to enter sleep mode and only be activated during motion detection. This configuration allows for lower power consumption.

\subsubsection{Software Architecture}

In \textbf{Figure \ref{fig:softwareArchitecture}}, the Notification Service is responsible for
processing and sending notifications to users when the motion is detected using different
means. In the current configuration, an SMS gateway using Twilio API was used for
communication with Client Devices. It is worth noting that the Twilio API is not the only API
that can be used for the system, as explained in the \textbf{Conclusion and Future Work}.

A database is required to store pertinent information regarding the user and the hardware. The
Reporting and Analytics component is responsible for generating reports on system performance,
such as the number of motion detections per day and average response time. In this
configuration, the Reporting and Analytics service held the tailored Jest and JMeter testing
suites and are further discussed in \textbf{Testing and Experimentation}.

\begin{figure}
    \centering
    \begin{adjustbox}{width=\linewidth}
        \begin{tikzpicture}
            \begin{umlcomponent}[x=4,y=0]{Cloud Workspace}
                \umlbasiccomponent[x=1,y=-2]{User Management Service - UI}
                \umlbasiccomponent[x=1,y=-4.5]{Reporting/Analytics Service}
                \umlbasiccomponent[x=1,y=-6.8]{Database}
                \begin{umlcomponent}[x=5.5,y=-3]{Notification Service}
                    \umlbasiccomponent[x=1, y=-2]{Twilio API}
                \end{umlcomponent}
            \end{umlcomponent}
            \umlbasiccomponent[x=4.5,y=-10.6]{Sensor Management Software}
            \umlbasiccomponent[x=11,y=-10.6]{Client Device}

            \draw[-latex, line width=1.5pt, dashed] (Sensor Management Software.north) -- (Notification Service.west);
            \draw[-latex, line width=1.5pt, dashed] (Twilio API.south) -- (Client Device.north);
        \end{tikzpicture}
    \end{adjustbox}
    \caption{Software Architecture}
    \label{fig:softwareArchitecture}
\end{figure}

The User Management Service (UI) provides an interface for users to register and
authenticate their devices and configure them. Users will then be able to change event logs or
otherwise view different metrics pertaining to system health and performance. The depicted
system were powered by Amazon AWS, but could be instead powered by other systems such as NoSQL
and common RDBMS solutions.

The Sensor Management Software is the software that directs the microcontroller and overall hardware assembly to manage power, send communications, and configure the microwave motion sensor. The user can configure the hardware module through wired or wireless communications, with the help of the web interface.

\begin{figure}
    \begin{adjustbox}{width=\linewidth}
        \begin{sequencediagram}
            \tikzstyle{inststyle}+=[bottom color=yellow]
            \tikzstyle{every node}+=[node distance=0.75cm and 0.75cm]

            \newthread{U}{:User}
            \newinst[0.75]{S}{:Sensor}
            \newinst[0.75]{T}{:Twilio}
            \newinst[0.75]{D}{:Database}

            \begin{sdblock}[green!20]{Device}{}
                \begin{call}{U}{Create Device}{S}
                {{\parbox{2cm}{\centering Ack Device Creation}}}
                    \begin{call}{S}
                    {Store Device Data}{D}{Ack Device Creation}
                    \end{call}
                \end{call}
            \end{sdblock}

            \begin{sdblock}[green!20]{Connect Client}{}
                \begin{call}{U}
                {{\parbox{2cm}{\centering Connect to Phone}}}{S}
                {{\parbox{2cm}{\centering Ack Connection Update}}}
                    \begin{call}{S}{Update Connection Status}{D}{Ack Connection Update}
                    \end{call}
                \end{call}
            \end{sdblock}

            \begin{sdblock}[green!20]{Subscribe to Alert}{}
                \begin{call}{U}
                {{\parbox{2cm}{\centering Subscribe to Alerts}}}{T}
                {{\parbox{2cm}{\centering Provide User's Alert Preferences}}}
                    \begin{call}{T}
                    {{\parbox{2cm}{\centering Retrieve User's Alert Preferences}}}{D}
                    {{\parbox{3cm}{\centering Ack Subs}}}
                    \end{call}
                \end{call}
            \end{sdblock}

            \begin{sdblock}[green!20]{Receive Alert}{}
                \begin{call}{U}{Request Alert}{S}{Notify User}
                    \begin{call}{S}{Retrieve Alert Data}{D}{Provide Alert Data}
                    \end{call}
                    \begin{call}{S}{Send Alert}{T}{}
                    \end{call}
                \end{call}
            \end{sdblock}

        \end{sequencediagram}
    \end{adjustbox}
    \caption{Software Sequence Diagram}
    \label{fig:softwareSeqDiagramUpdated}
\end{figure}

In \textbf{Figure \ref{fig:softwareSeqDiagramUpdated}}, a sequence diagram on multiple ends is depicted which shows a typical workflow for different actions. The actions involved include registering a device to a user's ecosystem, connecting an end client device, subscribing to hardware-provided alerts, and receiving an alert from the hardware. With the help of further documentation from the GitHub repository, more detail can be provided that the sequence diagram cannot account for..\cite{MorteSense-2023}

\subsection{Design Constraints, Problems, Trade-offs, and Solutions}\label{subsec:design-constraints-problems-trade-offs-and-solutions}

\subsubsection{Design Constraints and Challenges}

Designing a DIY home security system presents several constraints and challenges that
require careful consideration. The primary requirement for the system is that it should be
easy to use, install, and work seamlessly across a diverse range of devices, which
necessitates careful consideration of the user interface and communication protocols.
Additionally, the system must be robust and secure to ensure that it is not easily
hacked or compromised. This requires implementing encryption and secure data storage methods.
Furthermore, the system should have low power consumption, which necessitates the use of sleep
modes and energy-efficient components.

Lastly, the design must account for trade-offs, such as balancing the range and sensitivity of
the motion sensor with the overall system cost and size. For example,the dimensions of the
hardware enclosure should facilitate mounting the module on the doorstep or on the wall at an
angle that can provide a coverage angle between a typical visual coverage range of 60 to 75
degrees. However, the dimensions of the hardware enclosure should not be too tall or too wide,
as it may cause unequal weight distribution after mounting the enclosure.

\subsubsection{Design Solutions and Trade-offs}

An alternative design that was explored involved using a different power source, such as a
higher-voltage battery or a lower-voltage C/D battery, compared to the current 9V design.
However, using a lower voltage battery like AA/AAA would result in less effective mAh and
shorter battery life for the system. When using batteries like C/D, another challenge is the
additional space it would require in the enclosure, making it heavier and more difficult to
mount on walls or ceilings. For this project,9V batteries offer the best compromise, as they
have a reasonable form factor in the shape of a rectangular enclosure with rounded edges,
making it easy to design a 3D-printed enclosure with the help of CAD software.

To address the constraints and challenges of designing a DIY home security system, several
solutions and trade-offs have been incorporated into the system. The user interface design
aims to make it intuitive and user-friendly for users with different technical proficiency
levels. Compatibility is achieved by using widely adopted communication protocols and a
microcontroller with an integrated Wi-Fi module. To ensure robust security, the system employs
contemporary encryption techniques and secure data storage methods, including a secure
database hosted on Amazon AWS. In terms of energy efficiency, the device enters sleep mode
when not actively detecting motion, reducing power consumption. The system also includes
trade-offs,such as optimizing the motion sensor's range and sensitivity, which may impact the
overall cost and size of the system. Ultimately, these design solutions and trade-offs
contribute to a balanced, effective DIY home security system.
