\section{Conclusion and Future Work}\label{sec:conclusion-and-future-work}

The \brand{} DIY security system represents a significant step forward in providing
accessible and reliable security solutions for both residential and commercial properties.
By integrating microwave motion sensors and mobile notifications using a flexible implemenation,
the system offers an affordable and user-friendly approach to monitoring and securing properties
with ease of repair and maintenance in mind.

The design approach to leveraging contemporary microcontrollers and Wi-Fi connectivity, along
with its modular design principles, underscores the system's adaptability and potential for
future expansion. Furthermore, the integration of a user-friendly web interface and the ability
to incorporate different tools and services such as Twilio and Amazon AWS, without being limited to just those particular ecosystems, contribute to the overall accessibility and sustainability of the design. By considering factors such as user behavior, power outages,and software compatibility, the system was designed to mitigate these risks and ensure a robust security solution for end-users.

The comprehensive documentation, including the PCB diagram and Class Diagram, highlights the
meticulous planning and execution involved in the hardware and software integration. Future work
may involve exploring other tools and services such as FastAPI, the Telegram Bot, Google
Firebase Cloud Messaging, PostgreSQL, MySQL, NoSQL, Cloudflare, etc. Additionally, the
incorporation of advanced technologies such as blockchain and machine learning applications such
as computer vision, present promising avenues for further research and development in the field
of smart home security systems.

In summary, the \brand{} DIY security system's innovative use of technology, combined with
its emphasis on accessibility and user-friendliness, makes it a valuable and practical
solution for those seeking openly-documented security measures for their homes and businesses.
